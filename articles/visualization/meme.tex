\title{Visualization Exercises}
\author{hmys}
\date{14 07 2022}
\begin{document}
\maketitle

\par
This is an article that will list a number of visualization exercises that I have found useful in improving my own visualization abillities. It is a work in progress and I will continually update it as I find more interesting exercises. Some of them are best done after you have meditated, where your mind is in a very high degree of focus, some of them can be done as their own form of meditation.
\par
\subsection{Lines and Lines and Lines} The first exercise is the simplest one. You visualize n black equally long lines on a white background. A good number to start with is two. Then you visualize all possible permutations of 2-dimensional symbols that can be created. For example, with two lines you can create an L-shape, an x, a plus sign, a somewhat oddly proportioned T, a V, an A without the middle line, and a few more. When this gets every easy (or when you have exhaused all the permutations), you repeat the same process with one more line. With three lines you can create an equilateral triangle, and H, an ugly A, a "not equals" sign, a very angular U, a pi symbol, a k, a fat I, a Z, you can place the three lines connected only in one dot, so you get three lines pointing outwards like a crosshair. And so on and so on. 
\par
\subsection{Image-Streaming}
Close your eyes, and image some scenery, then pick something in the scenerey that stands out to you, then imagine moving towards it, picking it up, looking at it. Then turn around and imagine moving somewhere. Try to continually be moving, and to continually imagine new scenery for as long as you can. To get yourself going you might want to start by repeating some descriptions and words internally. An example: "sunny sky beach, ocean, southern europe". I imagine I am at a beach. There are beach balls, at it is very sunny. I turn around. There is a white house made of some kind of sandstone. It has a singular oaken door. I walk into the door. I am in a empty hallways that looks liminal. A woman with brown hair walks past me. I enter the corridor, at the end there is another door I open it. I am in an empty cafe that looks old with some windows where sun shines through. I walk through the cafe. There are tons of chars laying around, there are no people and no non-natural light. It looks abandoned. I walk to the window. It is a backstreet, looks like it could be in Northern Italy. I climb out. I look to the right, the ground is made of old-looking stones. The street is about three meters wide, and the buildings on both sides are very tall, probably four stories or so and have red brick roofs........ And so it goes for eternity.
\par
\subsection{Sharpening Intuitions of Mental Space}
Imagine a rectangular grid made up of unit-squares. Then imagine a rectangle whose first coordinate A is in (0,0), whose second coordinate B is on the x-axis and whose third coordinate C is on the y-axis. Then imagine rotating the line that stretches from C to B clockwise while the point at C is fixed, until the line lays on the y-axis. Then count how many units accross the origo the line falls. This is a way to mentally compute the pythagorean theorem. I always found it very strange that I can visualize a right triangle perfectly clearly in my mind, but that when I do this exercise, I have a very hard time reading off the length of the hypotenuse this way. Usually it is off by 30 percent or more. If I imagine a triangle A = (0,0), B = (1,0), and C =(0,1), and I rotate BC in the prescribed manned, its tip falls about 1/2 of a unit past the x-axis. So $\sqrt{2}$ should be ~ 1.5. This is actually not too bad, but for bigger number it becomes exceedingly difficult. You can also use this technique to compute cos and sin, but it is (I find) much much harder.

\end{document}
