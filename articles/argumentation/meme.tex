\documentclass[12pt,english]{article}
\usepackage{amsmath}
\usepackage{graphicx}
\graphicspath{{./}}

\title{Dishonest or Defensive Discourse Patterns}
\author{Levi Finkelstein}
\date{\today}

\begin{document}
\maketitle
\\\\
\it{This is a list in progress of discourse related behaviors that frequently irk me when I'm talking to people. This can be anything from clear dishonesty, to more unconscious dishonesty resulting from cognitive dissonance, to small  psychological defensive ticks, to vague uninformative unclear statements, to anything else that annoys me.}

\section*{Hedging} Something added to lessen commitment, either by making it unclear what’s being committed to, make the degree of commitment unclear, change the type of commitment (that it’s just some sort of personal opinion “just my opinion/I’m not an expert”). In general making what you’re saying weaker so it’s more difficult to attack. “I was *just* saying…”.

\section*{Crux Allergy} People will often flinch every time a crux is put on the table. They’ll claim something until a testable or more concrete proposition that can be looked up is identified that their claim would hinge on (the crux), then they’ll immediately walk back or weaken/make more uncertain what they’ve previously claimed. “If X was true, would that change your mind on Y?”

\section*{Ironic Conceit} Especially defensive people will often agree to something ironically (especially when conceding) to gain plausible deniability or to pretend they didn’t ever take the discussion seriously. “Yes, you were SOO right, I’m SOO wrong about this.” Dipping their toes into “I was just trolling all laong man XD '' territory to see if it’s plausible strategy.

\section*{Motte and Bailey} Before being challenged: uttering something that when interpreted on its face seems to obviously mean something very strong and controversial which can be easily attacked, then when challenged: explains that what they meant by the initial utterance was just something very weak and uncontroversial. When challenger leaves they return to the original strong sounding utterance. 

\section*{Distancing Language} Phrasing that psychologically distances you from your statement. Euphemism and hedging can be used for this purpose. Typical is the avoidance of first person pronouns, passive voicing (a mistake was made), using labels that give false implications “that woman”. A nice concrete sign that someone is uncomfortable with being explicit.

\section*{Euphemism} Expression that makes an utterance seem more innocuous, often as a way to emotionally distance yourself from what’s being talked about. The topic of euphemism is related to the discussion of when the emotional valence of your language fits what you’re talking about. “Animal holocaust.” 

\section*{Distinction without a Difference} People will often not agree to a characterization you give, then give their characterization only for it not to contain any distinction that actually makes a difference to what’s being talked about. This will reveal whether someone is instinctively disagreeing with what you’re saying which is evidence of dishonesty. It also occurs when they want to reframe your characterization in more euphemistic language. It also occurs when people want to not be lecured, they want to pretend like they're adding something to what's being said so they don't have to feel inferior.

\section*{Question Dodging} ...\\Like when people can't even answer the most uncontroversial easiest questions, there's obviously some extreme dishonesty or cognitive dissonance at play.

\section*{Fully General Analogy Rejection} ...
\end{document}
